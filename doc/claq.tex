\documentclass[letterpaper,11pt]{article}
\usepackage[scale=0.8]{geometry}
\usepackage{amsmath}
\usepackage{txfonts}
\usepackage[T1]{fontenc}
\usepackage{textcomp}
\usepackage{mathtools}

\DeclarePairedDelimiter{\ket}{\lvert}{\rangle}

\newcommand{\param}[1]{\ensuremath{\langle\text{\textit{#1}}\rangle}}

\title{Claq: Classical-to-quantum circuit synthesizer}
\author{Tsuyoshi Ito \\
  NEC Laboratories America, Inc.}
\date{}

\begin{document}
\maketitle

Claq is a tool which converts a classical combinational circuit to a quantum circuit.

\section{Usage}

Claq is invoked by the following command line:
\begin{quote}
\texttt{claq} [\param{option}\ldots] \param{input}\ldots
\end{quote}

Each \param{input} is the name of a file containing the description
of a classical combinational circuit in the format described
in Section~\ref{section:input-spec}.
For each \param{filename} on the command line,
Claq produces its report to the standard output.
All \param{option}s apply to all \param{input}s on the command line,
and the relative order of \param{option}s and \param{input}s is irrelevant.

Valid \param{option}s are as follows.
\begin{itemize}
\item
  \texttt{-{-}report=}\param{report-type}:
  This option specifies the type of the report that Claq produces.
  \param{report-type} is one of the following.
  \begin{itemize}
  \item
    \texttt{circuit} (default):
    The .QC circuit which consists of
    NOT, CNOT, and Toffoli gates.
    CNOT and Toffoli gates may have positive and negative controls.
  \item
    \texttt{circuitstd}:
    The .QC circuit which consists of
    gates in the standard gate set (Clifford and T gates).
  \item
    \texttt{raw}:
    The circuit in the internal format
    (primarily for debugging).
  \end{itemize}
\item
  \texttt{-s}:
  A shorthand for \texttt{-{-}report=circuitstd}.
\item
  \texttt{-n}, \texttt{-{-}noncoherent}:
  By default, Claq produces a quantum circuit
  which implements the specified classical function~$f$ in a coherent manner:
  $\ket{x}\ket{y}\ket{0}\mapsto\ket{x}\ket{y\oplus f(x)}\ket{0}$.
  When~\texttt{-n} is specified,
  Claq produces a quantum circuit
  which implements~$f$ in a noncoherent manner:
  $\ket{x}\ket{y}\ket{0}\mapsto\ket{x}\ket{y\oplus f(x)}\ket{\varphi_x}$,
  where~$\ket{\varphi_x}$ is some quantum state depending on~$x$.
  Specify~\texttt{-n} when you know that the noncoherent version is sufficient
  and want a smaller circuit.
\end{itemize}

\section{Input specification} \label{section:input-spec}

Claq reads classical circuits from files specified on the command line.
Each file contains one classical circuit, whose grammar will be specified in this section.

An example of a simple classical circuit is:
\begin{quote}
\begin{verbatim}
-- 2-bit adder with carry
.inputs a0, a1, b0, b1, c0;
.outputs s0, s1, s2;
s0 = a0 ^ b0 ^ c0;
c1 = a0 & (b0 | c0) | b0 & c0;
s1 = a1 ^ b1 ^ c1;
s2 = a1 & (b1 | c1) | b1 & c1;
\end{verbatim}
\end{quote}

\subsection{Lexical structure}

The lexical structure of an input file
is similar to that of Haskell 2010 with the notable exception
that identifiers can start or contain a period (``\verb|.|'').

A comment either begins with two consecutive dashes (``\verb|--|'')
and extends to the next newline,
or begins with the sequence of an open brace and a dash (``\verb|{-|'')
and ends with the sequence of a dash and a close brace (``\verb|-}|'').
Comments of the latter kind can be nested.

\subsection{Syntax}

The following is the syntax for the input file after lexical analysis
in a variation of the Backus--Naur form.
A vertical bar (``~|~'') means a choice,
brackets (``[\ldots]'') mean an optional part,
and braces (``\{\ldots\}'') mean zero or more repetitions.
\begin{align*}
  \param{input-file} ::=\; & \{\param{statement}\} \\[0.5\baselineskip]
%
  \param{statement} ::=\; & \param{inputs-statement} \\
  |\; & \param{outputs-statement} \\
  |\; & \param{equation-statement} \\
  |\; & \param{empty-statement} \\
  \param{inputs-statement} ::=\; & \text{\texttt{.inputs}}\;[\param{wire-name} \; \{\text{\texttt{,}}\;\param{wire-name}\}]\;\text{\texttt{;}} \\
  \param{outputs-statement} ::=\; & \text{\texttt{.outputs}}\;[\param{expr} \; \{\text{\texttt{,}}\;\param{expr}\}\;\text{\texttt{;}} \\
  \param{equation-statement} ::=\; & \param{wire-name}\;\text{\texttt{=}}\;\param{expr}\;\text{\texttt{;}} \\
  \param{empty-statement} ::=\; & \text{\texttt{;}} \\[0.5\baselineskip]
%
  \param{expr} ::=\; & \texttt{0} \\
  |\; & \texttt{1} \\
  |\; & \param{wire-name} \\
  |\; & \text{\texttt{(}}\;\param{expr}\;\text{\texttt{)}} \\
  |\; & \text{\texttt{\~{}}}\;\param{expr} \\
  |\; & \param{expr}\;\text{\texttt{\&}}\;\param{expr} \\
  |\; & \param{expr}\;\text{\texttt{\^}}\;\param{expr} \\
  |\; & \param{expr}\;\text{\texttt{|}}\;\param{expr} \\[0.5\baselineskip]
%
  \param{wire-name} ::=\; & \param{identifier which is not a keyword} \\
  \param{keyword} ::=\; & \text{\texttt{.inputs}} \;|\; \text{\texttt{.outputs}}
\end{align*}

Keywords and identifiers are case-sensitive.

In the production rules for \param{expr},
the unary operator \verb|~| has the highest precedence,
the binary operator \verb|&| has the second highest,
the binary operator \verb|^| has the third highest,
and the binary operator \verb#|# has the lowest precedence.
The associativity of the binary operators \verb|&|, \verb|^|, and \verb#|#
does not matter because of their semantics.

A valid input file must contain exactly one inputs statement
and exactly one outputs statement.
An input file may contain any number of equation statements and empty statements.
Empty statements do not have any meaning and completely ignored.
The order of equation statements does not have any meaning,
nor does the relative order of the inputs statement, the outputs statement,
and equation statements.

\subsection{Wire definitions}

All the wires that appear in expressions
(in the outputs statement or the right-hand side of an equation statement)
must be defined.

A wire is considered to be defined if it appears in the inputs statement
or it appears on the left-hand side of an equation statement.

No wires may be defined more than once.

\section{To-dos}

\begin{itemize}
\item
  Detect unused wires in input circuits and remove them.
\item
  Detect wires which are guaranteed to contain the same value
  (or the opposite values) in input circuits
  and merge them into one.
\item
  Implement any nontrivial optimizations.
\end{itemize}

\section*{About license of Claq}

Copyright \textcopyright\ 2013 NEC Laboratories America, Inc.
All rights reserved.

Redistribution and use in source and binary forms, with or without
modification, are permitted provided that the following conditions are
met:
\begin{itemize}
\item
  Redistributions of source code must retain the above copyright
  notice, this list of conditions and the following disclaimer.
\item
  Redistributions in binary form must reproduce the above
  copyright notice, this list of conditions and the following
  disclaimer in the documentation and/or other materials provided
  with the distribution.
\item
  Neither the name of NEC Laboratories America, Inc.\ nor the names
  of its contributors may be used to endorse or promote products
  derived from this software without specific prior written
  permission.
\end{itemize}

THIS SOFTWARE IS PROVIDED BY THE COPYRIGHT HOLDERS AND CONTRIBUTORS
``AS IS'' AND ANY EXPRESS OR IMPLIED WARRANTIES\@, INCLUDING\@, BUT NOT
LIMITED TO\@, THE IMPLIED WARRANTIES OF MERCHANTABILITY AND FITNESS FOR
A PARTICULAR PURPOSE ARE DISCLAIMED\@.  IN NO EVENT SHALL THE COPYRIGHT
OWNER OR CONTRIBUTORS BE LIABLE FOR ANY DIRECT\@, INDIRECT\@, INCIDENTAL\@,
SPECIAL\@, EXEMPLARY\@, OR CONSEQUENTIAL DAMAGES (INCLUDING\@, BUT NOT
LIMITED TO\@, PROCUREMENT OF SUBSTITUTE GOODS OR SERVICES\@; LOSS OF USE\@,
DATA\@, OR PROFITS\@; OR BUSINESS INTERRUPTION) HOWEVER CAUSED AND ON ANY
THEORY OF LIABILITY\@, WHETHER IN CONTRACT\@, STRICT LIABILITY\@, OR TORT
(INCLUDING NEGLIGENCE OR OTHERWISE) ARISING IN ANY WAY OUT OF THE USE
OF THIS SOFTWARE\@, EVEN IF ADVISED OF THE POSSIBILITY OF SUCH DAMAGE.

\section*{Acknowledgments}

The author thanks Martin Roetteler for helpful discussions.
The development of Claq was supported in part by the Quantum Computer Science Program
of the Intelligence Advanced Research Projects Activity.

\end{document}
